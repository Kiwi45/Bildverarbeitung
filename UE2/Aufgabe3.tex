% das Papierformat zuerst
\documentclass[a4paper, 12pt]{article}
\linespread{1.15}
\usepackage{verbatim}
%reduce whitespaces
\usepackage[margin=1in]{geometry}
\usepackage{graphicx} 
\usepackage{float}
\usepackage{booktabs}
\usepackage[svgnames]{xcolor}
\usepackage{listings}
\lstset{breaklines}
\usepackage{afterpage}

%adjust footer/ header
\usepackage{fancyhdr}
\pagestyle{fancy}
\fancyhf{}

\renewcommand{\footrulewidth}{1pt}
\renewcommand{\headrulewidth}{0pt}
\fancyfoot{} % clear all footer fields
\fancyfoot[LE,RO]{\thepage}           % page number in "outer" position of footer line
\fancyfoot[RE,LO]{Carolin Kunze(552038), Kevin Noessler} 

%add Code
\usepackage{listings}

\title{SV4}

% hier beginnt das Dokument
\begin{document}
\maketitle
\thispagestyle{fancy}

Nachtrag zu Aufgabe 2\\
e) Wenn man das Signal als Audiosignal betrachtet klingt es mehrfach aneinandergereiht wie ein Alarmsignal.\\


Aufgabe 3\\
In dieser Aufgabe multiplizieren wir zun{\"a}chst die 3 Faktoren: das allgemeine Zirpsignal aus Aufgabe 1d, die Kosinusglocke als Fensterfunktion und die komplexe Exponentialfunktion.
Anschließend intergrieren wir von 0 bis 4 Sekunden mit der Trapezregel in Matlab, indem wir den Vektor x mit den Werten von 0 bis 4 Sekunden und die 5 Produkte der 3 genannten Faktoren zu den jeweiligen Zeitpunkten {\"u}bergeben.

Die Kosinusglocke berechnen wir mit folgender Formel:


\begin{center}
	\begin{equation}
    
                \begin{array}{ll}
                 g(t) = 1 + cos(\pi*\alpha*t)
                \end{array}
              
	\end{equation}
\end{center}


Da die Fensterbreite $\alpha/2 = 0,4$ Sekunden gegeben ist, k{\"o}nnen wir $\alpha = 5$ Sekunden umstellen und weiter einsetzen.

Die Frequenz f haben wir mit 15625 Hz berechnet, sodass wir $\omega$ mit $\omega = 2*\pi*f$ f{\"u}r die Umstellung der komplexen Exponentialfunktion berechnen k{\"o}nnen.


\begin{center}
	\begin{equation}
   
                \begin{array}{ll}
                 e^{-j*\omega*t} = \omega*cos(t)+(-j)sin(t)
                \end{array}
             
	\end{equation}
\end{center}


Das Signal berechnet sich wie folgt:

\begin{center}
	\begin{equation}
   
                \begin{array}{ll}
                 signal = A * sin(\pi*f1*t+2*\pi*f0*t)
                \end{array}
             
	\end{equation}
\end{center}


Das Produkt aus den 3 Faktoren l{\"a}sst sich folgendermaßen darstellen:

\begin{center}
	\begin{equation}
   
                \begin{array}{ll}
                 product = signal*(1+cos(\pi*5*t*(2*\pi*15625*cos(t-(\sqrt{-1}*sin(t)))
                \end{array}
             
	\end{equation}
\end{center}

Da wir uns noch unsicher über die Ergebnisdarstellung unserer Berechnung sind, haben wir leider keine Ergebnisdiagramme.



\end{document}