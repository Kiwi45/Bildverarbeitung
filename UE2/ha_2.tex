% (c) Gerard Baecker
\documentclass[fleqn,a4paper,12pt]{article}
\usepackage[german]{babel}
\usepackage{amsmath}    % Mathematische Symbole
\usepackage{amssymb}     % Nochmehr mathematische Symbole
\usepackage{dsfont}      % Schriftsatz fuer Zahlenmengensymbole
%\usepackage{verbatim}   % erweiterte Verbatim-Umgebung
\usepackage{alltt}       % Quasi-Verbatim-Umgebung
\usepackage{fancyhdr}    % Eigene Kopfzeilen
\usepackage{graphicx}    % Zum Einbinden von Grafiken
% Einbinden einer eps-Grafik geht so: includegraphics{path}

% Seitenraender
\addtolength{\voffset}{-2cm}
\addtolength{\textheight}{0cm}
\addtolength{\hoffset}{0cm}
\addtolength{\textwidth}{2cm}
\addtolength{\headheight}{2cm} % fuer jeden Strichkode einen Zentimeter

% Skalierung der Grafiken
\setlength{\unitlength}{1cm}

\pagestyle{fancy}            % Eigene Kopfzeilen verwenden
\frenchspacing               % Kein Extrafreiraum nach Satzzeichen
\setlength{\parindent}{0pt}  % Neue Absaetze nicht einruecken
%\sloppy                     % Schlampige Absatzformatierung
\fussy                       % Penible Absatzformatierung
\linespread{1.5}             % Zeilenabstand

% Font fuer Code 39
\font\xlix=wlc39 scaled 1200
\newcommand\barcode[1]{{\xlix@#1@}}

% Name, Matrikelnummer, Barcode
\newcommand\student[2]{
  \mbox{\scriptsize
  \begin{tabular}{@{}l@{}r@{}}
    \multicolumn{2}{@{}r@{}}{\barcode{#2}}\\
    #1&#2\\
  \end{tabular}}}

% Kopfzeile
\lhead{
  \small
  \textsc{Spezialgebiete der Bildverarbeitung \\
    SS 2018 \\
    \"Ubung (08.05.2018)}
  \vfill}
\rhead{
  \begin{tabular}[b]{@{}rr@{}}
    \student{Carolin Kunze}{*552038*} &\\
     \student{Kevin Noessler}{*569626*}  & \\   
  \end{tabular}}

\begin{document}

"Ubungsaufgabe 2: \newline

a) Im ersten Graphen haben wir mit 1024 Messwerten im zeitlichen Abstand von jeweils 8 Millisekunden ein einfaches Zirpsignal generiert.
Man kann beaochten, dass das Signal bis 4 Sekunden eine höhere Frequenz hat und diese dann sehr stark sinkt und wieder etwas ansteigt und das Signal mit niedrigerer Frequenz bis kurz nach 8 Sekunden weiterl{\"a}uft.

\begin{center}
\begin{figure}[h]
    \begin{scriptsize}
  \begin{center}
      \includegraphics[scale = 0.65]{graph_2a.png}
      \end{center}
 	\end{scriptsize}
  \end{figure}
\end{center}
  
b) Die 5 Kennwerte in Gleichung 2.7 des Arbeitsblattes kann man folgendermaßen bestimmen:
da es sich um einfaches Zirpsignal handelt, wissen wir, dass f0 gleich Null ist und f1 ungleich Null, d.h. eine Frequenz{\"a}nderung besitzt.
Dies konnten schon wir schon in der Aufgabe 2a) gut beaobachten. Die Amplitude A verändert sich während der Zeit ein wenig und ist auch nicht kosntant.
Sie liegt bis 4 Sekunden approximiert bei 1,75 und ab 4 Sekunden dann fast konstant bei 2.
Das Zirpsignal hat keinen Gleichanteil s0 und auch keinen Phasenwinkel.\\



c) F{\"u}r die Pr{\"u}fung auf schwache Stationarit{\"a}t haben wir den Mittelwert, die Standardabweichung sowie die Varianz von jeweils 64 Messwerten am Signalanfang, in der Signalmitte sowie an dem Signalende berechnet.

\begin{table}[h]
\centering
\begin{tabular}{l|l|c|l}
             & \multicolumn{1}{c|}{Mittelwerte} & Standardabweichung & \multicolumn{1}{c}{Varianz} \\ \hline
Signalanfang & -0,070523438                     & 1,325318678        & 1,784350067                 \\
Signalmitte  & 0,121816923                      & 0,840983014        & 0,718303249                 \\
Signalende   & -0,078418462                     & 1,440280558        & 2,106820713                
\end{tabular}
\end{table}

Dabei waren die Werte am Signalanfang sowie am Signalende zum Teil sehr ähnlich, jedoch die Werte in der Signalmitte sehr verschieden.
Dies kann man auch gut im Graphen aus 2a) beaobachten, da sich in der Signalmitte (bei 4 Sekunden) die Frequenz stark ändert.
Stationarit{\"a}t im Falle von Signalen bedeutet, dass ihre statistischen EIgenschaften nicht von der Zeit abh{\"a}ngig sind.
Stark stationäre Signale haben eine zeitunabh{\"a}ngige Verteilungsfunktion, schwach station{\"a}re Signale haben nur einen zeitunabhängigen Erwartungswert und Varianz. (Quelle: Werkzeuge der Signalverarbeitung)
Da die statistischen Werte unseres Zirpsignals jedoch zeitabh{\"a}ngig sind, ist das Signal instation{\"a}r, also auch nicht schwach station{\"a}r.

\newpage
d) Im nächsten Schritt betrachten wir das Signal spektral, indem wir für die kurzen Episoden am Signalanfang, in der Signalmitte und am Signalende die DFT-Spektren berechnen und die DFT-Betragsspektren darstellen.

\begin{center}
\begin{figure}[h]
    \begin{scriptsize}
  \begin{center}
      \includegraphics[scale = 0.65]{gra1.png}
      \end{center}
 	\end{scriptsize}
  \end{figure}
\end{center}

\begin{center}
\begin{figure}[h]
    \begin{scriptsize}
  \begin{center}
      \includegraphics[scale = 0.5]{gra2.png}
      \end{center}
 	\end{scriptsize}
  \end{figure}
\end{center}

\begin{center}
\begin{figure}[h]
    \begin{scriptsize}
  \begin{center}
      \includegraphics[scale = 0.65]{gra3.png}
      \end{center}
 	\end{scriptsize}
  \end{figure}
\end{center}

\newpage
Alle 3 Betragspektren haben einen {\"a}hnlichen Verlauf. Bei niedrigem und hohen k ist das Betragsspektrum hoch, anderfalls konstant niedrig. \\


e) Schlussfolgerung: das Signal ist nicht schwach station{\"a}r.





\end{document}