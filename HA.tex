% (c) Gerard Baecker
\documentclass[fleqn,a4paper,12pt]{article}
\usepackage[german]{babel}
%\usepackage{amsmath}    % Mathematische Symbole
\usepackage{amssymb}     % Nochmehr mathematische Symbole
\usepackage{dsfont}      % Schriftsatz fuer Zahlenmengensymbole
%\usepackage{verbatim}   % erweiterte Verbatim-Umgebung
\usepackage{alltt}       % Quasi-Verbatim-Umgebung
\usepackage{fancyhdr}    % Eigene Kopfzeilen
\usepackage{graphicx}    % Zum Einbinden von Grafiken
% Einbinden einer eps-Grafik geht so: includegraphics{path}

% Seitenraender
\addtolength{\voffset}{-2cm}
\addtolength{\textheight}{0cm}
\addtolength{\hoffset}{0cm}
\addtolength{\textwidth}{2cm}
\addtolength{\headheight}{2cm} % fuer jeden Strichkode einen Zentimeter

% Skalierung der Grafiken
\setlength{\unitlength}{1cm}

\pagestyle{fancy}            % Eigene Kopfzeilen verwenden
\frenchspacing               % Kein Extrafreiraum nach Satzzeichen
\setlength{\parindent}{0pt}  % Neue Absaetze nicht einruecken
%\sloppy                     % Schlampige Absatzformatierung
\fussy                       % Penible Absatzformatierung
\linespread{1.5}             % Zeilenabstand

% Font fuer Code 39
\font\xlix=wlc39 scaled 1200
\newcommand\barcode[1]{{\xlix@#1@}}

% Name, Matrikelnummer, Barcode
\newcommand\student[2]{
  \mbox{\scriptsize
  \begin{tabular}{@{}l@{}r@{}}
    \multicolumn{2}{@{}r@{}}{\barcode{#2}}\\
    #1&#2\\
  \end{tabular}}}

% Kopfzeile
\lhead{
  \small
  \textsc{Grundlagen der Signalverarbeitung \\
    SS 2018 \\
    \"Ubung (24.04.2018)}
  \vfill}
\rhead{
  \begin{tabular}[b]{@{}rr@{}}
    \student{Caroline Kunze}{*552038*} &\\
     \student{Kevin Noessler}{*569626*}  & \\
     %\student{Sedat }{*111111*}  &     
  \end{tabular}}

\begin{document}

"Ubungsaufgabe 1: \newline
1. Grafische Darstellung:
\begin{center}
\begin{figure}[h]
    \begin{scriptsize}
  \begin{center}
      \includegraphics[scale = 0.8]{HA1.png}
      \end{center}
 	\end{scriptsize}
  \end{figure}
\end{center}
  
\newpage

2. Beurteilung der Audiosignale (4 Sekunden)

\begin{table}[h]
\begin{tabular}{ll}
\hline
Signal                 & Beurteilung                                                                                                                        \\ \hline
Gleichsignal           & \begin{tabular}[c]{@{}l@{}}Trivialer Fall, da bei der Wiedergabe des \\ Audiosignal kein Ton zu h{\"o}ren ist.\end{tabular}            \\
&\\
Sinussignal            & \begin{tabular}[c]{@{}l@{}}In der Wiedergabe des Audiosignals wird \\ einen durchgehender, konstanter Ton erzeugt.\end{tabular} \\
&\\
Einfaches Zirpsignal   & \begin{tabular}[c]{@{}l@{}}In der Wiedergabe des Audiosignals wird \\ einen immer h{\"o}her werdender Ton erzeugt.\end{tabular}     \\
&\\
Allgemeines Zirpsignal & \begin{tabular}[c]{@{}l@{}}In der Wiedergabe des Audiosignals wird \\ ein fallender Ton erzeugt.\end{tabular}                   \\ \hline
\end{tabular}
\end{table}

\newpage
3. Was passiert, wenn Sie die Signale l{\"a}nger beobachten?

\begin{table}[h]
\begin{tabular}{ll}
\hline
Signal                 & Beurteilung                                                                                                                        \\ \hline
Gleichsignal           & \begin{tabular}[c]{@{}l@{}}Trivialer Fall, da bei der Wiedergabe des \\ Audiosignal kein Ton zu h{\"o}ren ist.\end{tabular}            \\
&\\
Sinussignal            & \begin{tabular}[c]{@{}l@{}}In der Wiedergabe des Audiosignals wird \\ einen durchgehender, konstanter Ton erzeugt, \\ der auch   nach mehr als 4 Sekunden konstant ist.\end{tabular} \\
&\\
Einfaches Zirpsignal   & \begin{tabular}[c]{@{}l@{}}In der Wiedergabe des Audiosignals wird einen  \\immer h{\"o}her werdender Ton erzeugt, der jedoch \\ wieder abf{\"a}llt im Laufe der Zeit\end{tabular}     \\
&\\
Allgemeines Zirpsignal & \begin{tabular}[c]{@{}l@{}}In der Wiedergabe des Audiosignals wird \\ ein fallender Ton erzeugt, der jedoch wieder ansteigt \\ im Laufe der Zeit\end{tabular}                   \\ \hline
\end{tabular}
\end{table}

\end{document}
