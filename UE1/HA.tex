% (c) Gerard Baecker
\documentclass[fleqn,a4paper,12pt]{article}
\usepackage[german]{babel}
%\usepackage{amsmath}    % Mathematische Symbole
\usepackage{amssymb}     % Nochmehr mathematische Symbole
\usepackage{dsfont}      % Schriftsatz fuer Zahlenmengensymbole
%\usepackage{verbatim}   % erweiterte Verbatim-Umgebung
\usepackage{alltt}       % Quasi-Verbatim-Umgebung
\usepackage{fancyhdr}    % Eigene Kopfzeilen
\usepackage{graphicx}    % Zum Einbinden von Grafiken
% Einbinden einer eps-Grafik geht so: includegraphics{path}

% Seitenraender
\addtolength{\voffset}{-2cm}
\addtolength{\textheight}{0cm}
\addtolength{\hoffset}{0cm}
\addtolength{\textwidth}{2cm}
\addtolength{\headheight}{2cm} % fuer jeden Strichkode einen Zentimeter

% Skalierung der Grafiken
\setlength{\unitlength}{1cm}

\pagestyle{fancy}            % Eigene Kopfzeilen verwenden
\frenchspacing               % Kein Extrafreiraum nach Satzzeichen
\setlength{\parindent}{0pt}  % Neue Absaetze nicht einruecken
%\sloppy                     % Schlampige Absatzformatierung
\fussy                       % Penible Absatzformatierung
\linespread{1.5}             % Zeilenabstand

% Font fuer Code 39
\font\xlix=wlc39 scaled 1200
\newcommand\barcode[1]{{\xlix@#1@}}

% Name, Matrikelnummer, Barcode
\newcommand\student[2]{
  \mbox{\scriptsize
  \begin{tabular}{@{}l@{}r@{}}
    \multicolumn{2}{@{}r@{}}{\barcode{#2}}\\
    #1&#2\\
  \end{tabular}}}

% Kopfzeile
\lhead{
  \small
  \textsc{Spezialgebiete der Bildverarbeitung \\
    SS 2018 \\
    \"Ubung (24.04.2018)}
  \vfill}
\rhead{
  \begin{tabular}[b]{@{}rr@{}}
    \student{Carolin Kunze}{*552038*} &\\
     \student{Kevin Noessler}{*569626*}  & \\
     %\student{Sedat }{*111111*}  &     
  \end{tabular}}

\begin{document}

"Ubungsaufgabe 1: \newline
1. Grafische Darstellung:
\begin{center}
\begin{figure}[h]
    \begin{scriptsize}
  \begin{center}
      \includegraphics[scale = 0.8]{HA1.png}
      \end{center}
 	\end{scriptsize}
  \end{figure}
\end{center}
  
\newpage

2. Beurteilung der Audiosignale (4 Sekunden)

\begin{table}[h]
\begin{tabular}{ll}
\hline
Signal                 & Beurteilung                                                                                                                        \\ \hline
Gleichsignal           & \begin{tabular}[c]{@{}l@{}} Da beim Gleichsignal die konstante Frequenz \\und die Frequenz{\"a}nderung gleich null sind, ist lediglich\\ eine Gerade zu beobachten. Bei der Wiedergabe\\ des Audiosignals ist somit auch kein Ton zu h{\"o}ren.\end{tabular}            \\
&\\
Sinussignal            & \begin{tabular}[c]{@{}l@{}} Das Sinussignal besteht aus Sinuskurven und durch die\\ konstante Frequenz von 1 kHz ist bei der Wiedergabe \\ des Audiosignals ein durchgehender, konstanter Ton\\ zu h{\"o}ren, welcher auf der gleichen Tonh{\"o}he bleibt.\end{tabular} \\
&\\
Einfaches Zirpsignal   & \begin{tabular}[c]{@{}l@{}} Das Einfache Zirpsignal hat anfangs eine niedrigere\\ Frequenz, welche sich im Verlauf der 4 Sekunden\\ durch eine positive Frequenz{\"a}nderung erh{\"o}ht. In der \\ Wiedergabe des Audiosignals ist \\dies durch einen Anstieg der Tonh{\"o}he zu beobachten. \end{tabular}     \\
&\\
Allgemeines Zirpsignal & \begin{tabular}[c]{@{}l@{}} Das Allgemeine Zirpsignal hat eine negative\\ Frequenz{\"a}nderung, wodurch sich die Frequenz im \\Verlauf der 4 Sekunden verringert. In der Wiedergabe des\\ Audiosignals wird ein fallender Ton erzeugt.\end{tabular}                   \\ \hline
\end{tabular}
\end{table}

\newpage
3. Was passiert, wenn Sie die Signale l{\"a}nger beobachten?

\begin{table}[h]
\begin{tabular}{ll}
\hline
Signal                 & Beurteilung                                                                                                                        \\ \hline
Gleichsignal           & \begin{tabular}[c]{@{}l@{}}Bei der Wiedergabe des Audiosignals ist auch \\nach 4 Sekunden weiterhin kein Ton zu h{\"o}ren, \\da sich die Frequenz sowie die Frequenz{\"a}nderung\\ nicht {\"a}ndert und diese Werte weiterhin Null sind.\end{tabular}            \\
&\\
Sinussignal            & \begin{tabular}[c]{@{}l@{}}In der Wiedergabe des Audiosignals wird weiterhin \\ein durchgehender, konstanter Ton erzeugt, der auch\\ nach mehr als 4 Sekunden seine Tonh{\"o}he nicht {\"a}ndert.\end{tabular} \\
&\\
Einfaches Zirpsignal   & \begin{tabular}[c]{@{}l@{}}In der Wiedergabe des Audiosignals wird ein immer\\ h{\"o}her werdender Ton erzeugt, der nach 4 \\Sekunden wieder f{\"a}llt. Dies liegt am Aliasing Effekt,\\ da die Abtastrate zu niedrig ist\\ (Quellen: http://www.asel.udel.edu/speech/tutorials/\\instrument/sam_rat.html, https://de.wikipedia.org/wiki/\\Alias-Effekt). Ab dem Zeitpunkt, in dem die Rate weniger als\\ doppelt so groß ist wie die Frequenz, f{\"a}llt\\ die Tonh{\"o}he. In unserem Fall ist dies ab \\4 Sekunden bei 4kHz, da die Abtastrate bei 8kHz liegt. \end{tabular}     \\
&\\
Allgemeines Zirpsignal & \begin{tabular}[c]{@{}l@{}}In der Wiedergabe des Audiosignals wird \\ ein fallender Ton erzeugt, der jedoch wieder ansteigt.\\ Nach 6 Sekunden ist eine Frequenz von Null erreicht und\\ somit ist ab einer niedrigen Frequenz kein Ton mehr\\ zu h{\"o}ren bevor der Ton wieder h{\"o}her wird, da die\\ Frequenz wieder steigt und nicht\\ negativ werden kann. \end{tabular}                   \\ \hline
\end{tabular}
\end{table}

\end{document}
